\documentclass{book}
\usepackage{color}
\usepackage{amsthm}
\usepackage{listings}
\lstset{
    basicstyle=\small \ttfamily,
    frame=lines,
    escapechar=`
}
\lstdefinelanguage{Coq}
    {morekeywords={Theorem,Proof,Qed,Definition,
        Fixpoint,Example,Lemma,match,end,if,else},
     sensitive=true,
     morecomment=[n]{(*}{*)},
     %commentstyle=\color{green}\scriptsize,
     commentstyle=\footnotesize,
     %commentstyle=\tiny,
     keywordstyle=\color{black}\bfseries,
     literate={forall}{{$\forall$}}1 {exists}{{$\exists$}}1 {~}{{$\neg$}}1
     {->}{{$\rightarrow$}}1 {/\\}{{$\wedge$}}1 {<->}{{$\iff$}}1
     {\\/}{{$\vee$}}1 {Qed}{{$\qedsymbol$}}1,
    }
\usepackage{ctex}
\usepackage{hyperref}
\begin{document}
\tableofcontents
\chapter{定理证明}
\section{tactics}
目前学过并且用了的tactics有:
\begin{itemize}
    \item 归纳法
    \item 运算化简(\verb|simpl|)
    \item 运用已知的结论、中间结论、前提,进行改写(\verb|rewrite|).
        这一步感觉是最难的,有几点需要弄清楚:
        \begin{itemize}
            \item 有那些结论可用?
            \item 改写哪一部分?
            \item 改写的时候,要避免free variable capture的问题
        \end{itemize}
    \item 分类讨论(\verb|destruct|)
\end{itemize}

还有一些方法是没有用到的,比如反证法, 比如逻辑连接词的使用。
比如怎么证明 $ 3 \ne 4 $\ ?
\begin{lstlisting}[language=Coq]
Theorem three_noteq_four: (3 =? 4) = false.
Proof.
    simpl. reflexivity.
Qed.
\end{lstlisting}
%`$\qedsymbol$`

\section{More tactics}
新学习了一些tactics,包括
\begin{itemize}
    \item discriminate
    \item injection
    \item apply \qquad 对有$\forall$限定的变量,进行替换。有\verb|with|变体
    \item \lstinline{generalize dependent}
\end{itemize}
然后,很多tactic都有\verb|in|变体.

有一些tactic只是用于简化证明,并没有引入什么新内容的,比如 \verb|symmetry|.

\subsection{generalize dependent}
需要先\verb|intros| 所有量词,然后再按照希望的顺序,重新泛化部分量词。
通过一个例子来说明。
\begin{lstlisting}[language=Coq]

Theorem nth_error_after_last: forall (n : nat) (X : Type) (l : list X),
    length l = n ->
    nth_error l n = None.
Proof.
    intros n X l.
    generalize dependent n.
    (* Why? We need to induction on l, but doing so,
       will first intros n automatically, and so n in the premise is not so
       general *)
    induction l as [| h t].
    - (* For empty list *)
      simpl. reflexivity.
    - (* For non empty list, l = x :: t *)
      simpl.
      induction n as [| n'].
      + (* n cann't be 0 *)
        simpl. intros eq1. discriminate eq1.
      + (* n = S n' *)
        simpl. intros eq2.
        injection eq2. intros eq3. apply IHt. apply eq3.
Qed.
\end{lstlisting}

再来一个例子
\begin{lstlisting}[language=Coq]%

Theorem implies_implies_and: forall P Q R : Prop,
    ((P -> Q) -> R) -> (P -> (Q /\ R).
\end{lstlisting}

这个结论其实是错的。考虑,如果\verb|P=True,Q=False|,那么
\lstinline[language=Coq]|(P -> Q) = False|,左边\lstinline[language=Coq]|False->R|恒为真。然而
\lstinline[language=Coq]|Q = False -> (Q /\ R) = False|,所以右边为假。因此结论不成立。

\chapter{关于学习}
\section{影响学习效果的因素}
\begin{itemize}
    \item 天赋
    \item 经验 \newline
        这里的经验是有效经验。如果一个经验重复了10000次,那么也只是一个有效经验。
        所以重复的、无意义的重复劳动,要尽量避免。哪怕是失败的经验,只要不是冗余的,也是宝贵的。
        另外,多读一些经典的书籍,也是快速积累有效经验的捷径。
        要经常自我反省,总结。
    \item 反馈 \newline
        要接受有效的反馈,需要
        \begin{itemize}
            \item 保持真诚开放的心态。
                不要因为情绪的因素,而选择性忽视事实,忽视反馈。
            \item 敏锐的观察,以及对效果的评估。能否将反馈量化/可视化?
            \item 考察重要的因子。
                有些影响反馈的因子,自己是很难察觉的,有点类似思维的盲点。这时候最好是跟别人
                交流,别人能帮你拓宽视野,指出一些关键的影响因子。
        \end{itemize}
\end{itemize}
人的精力是有限的,要么是选择做很多事,但是每件事都做的半吊子;要么选择做重要的事,宁可少做一点,
也要争取做好每件事。我选择后一种态度。

\section{能力}
\begin{itemize}
    \item 记忆
    \item 观察力 \newline
        有观察力,意味着能发现不同的行动会导致的微妙的变化,从而在需要精细操作的时候,能够领悟到需要做到什么地步。
        比如画家需要有出色的色觉,音乐家需要有出色的音感。
    \item 动手能力 \newline
        通过观察,知道需要怎么做,这还不足够,还要能确保落实做到。
    \item 精力 \newline
        一个人如果经常容易犯困,或者学一会儿就厌倦疲劳,那么就不能长时间投入地去学习。这对于一些复杂的理论,就难以掌握。
    \item 触类旁通 \newline
        举一反三,是不是一种能力?存疑。
\end{itemize}

通过比喻来说明的话,比如知道某分量的菜,需要放多少盐,这是观察力。

知道要放多少盐,但是在操作的时候,有没有手抖,对火候的控制能否做到随心所欲,这就是动手能力。

\section{如何提高能力}
天赋固然重要,但是天赋难以变改,除非以后科技发达,人可以比较自由的对自我进行元编程改造,那时候再来针对“天赋”
进行完善。

排除天赋的因素,这里考察人力所能改变的能力,进行探讨。
熟能生巧,多练习也可以改进各种能力。
以下罗列我认为可能有用的方法。
\begin{itemize}
    \item 记忆 \newline
        适当背诵。临睡/早起的时候,进行冥想回忆。
    \item 观察力
    \item 动手能力 \newline
        多练习,针对反馈进行调整。
    \item 精力 \newline
        注意锻炼身体。
        怎么进行有效的休息?睡觉前不要看太刺激的内容,保持平静的心境。可以考虑喝点牛奶
        \begin{itemize}
            \item 睡觉前不要看刺激兴奋的内容,保持平静的心境
            \item 考虑喝点牛奶
            \item 居住环境保持舒适干净
            \item \dots
        \end{itemize}
    \item 触类旁通 \newline
        需要经常的将新知识与旧经验建立联系。时不时对知识进行梳理,画下脑图,做点笔记。
\end{itemize}

\section{性格}
性格、爱好对学习也有影响。
如果某样技艺,需要花长年累月的练习,才能练好,那么就需要练习者对此有强烈的兴趣和爱好,
否则是很难坚持得下来的。
这里对兴趣爱好的讨论太宽泛了,
如果具体化,比如,喜欢小说电影,喜欢游山玩水,喜欢锻炼,喜欢天马行空的想象,喜欢创作,
这些喜好对人的影响是怎样的呢,对学习的影响又是怎样的呢。
另, 爱好太宽泛,就不宜深入。
另外还要能抵挡分心的诱惑。比如,一个人能不能坐得住,能不能耐住寂寞。一个人要能克制。

野心,激情,傲慢,这些非理性的因素,对学习的影响是如何的呢?

\section{对性格的改造}

\end{document}
